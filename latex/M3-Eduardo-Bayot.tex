\documentclass[a4paper,12pt]{article}
\usepackage{amsmath}
\usepackage{graphicx}
\usepackage{hyperref}
\usepackage{biblatex}

\addbibresource{M3-Eduardo-Bayot.bib} % Archivo .bib para las referencias


\begin{document}

% Carátula
\begin{titlepage}
    \begin{center}
        \includegraphics[width=0.3\textwidth]{logo.png}\\[1cm] % Inserta el logo aquí, ajusta la ruta si es necesario
        \textsc{\LARGE Universidad Siglo 21}\\[1.5cm]
        \textsc{\Large Trabajo Práctico Nro 3}\\[0.5cm]
        \textsc{\large Estadística y Probabilidad}\\[2cm]

        \rule{\linewidth}{0.5mm} \\[0.4cm]
        {\huge \bfseries Validación de Hipótesis Estadísticas}\\[0.4cm]
        \rule{\linewidth}{0.5mm} \\[1.5cm]

        \begin{tabular}{rl}
            \small Profesor Titular Experto: & \small Yanina Nancy Morales \\
            \small Profesor Titular Disciplinar: & \small Horacio José Caballero \\
        \end{tabular}
        \\[1.5cm]

        \textbf{\small Estudiante: Eduardo Bayot}\\
        \textbf{\small Cátedra: D}\\
        \textbf{\small Grupo: 1}
    \end{center}
\end{titlepage}

% Reiniciar numeración después de la carátula
\pagenumbering{arabic}


% Configuración de la página
\date{}
\author{}

\tableofcontents




\tableofcontents

\section{Introducción}
Breve contexto del problema planteado y objetivos del análisis.

\section{Metodología}
\subsection{Planteo del Problema}
Definición de hipótesis nula y alternativa.
\subsection{Simulación de Datos en R}
Explicación del uso de R para generar la muestra aleatoria.
\subsection{Prueba de Hipótesis}
Justificación del enfoque estadístico y descripción de los pasos realizados.

\section{Desarrollo}
\subsection{Tabla de Valores Simulados}
Presentación de los datos generados.
\subsection{Planteo de Hipótesis}
Definición formal de \( H_0 \) y \( H_a \).
\subsection{Cálculo del Estadístico de Prueba}
Fórmula utilizada y valor calculado.
\subsection{Determinación de la Región de Rechazo}
Cálculo del valor crítico y comparación con el estadístico.
\subsection{Decisión y Conclusión}
Discusión de los resultados y decisión respecto a \( H_0 \).

\section{Resultados}
Interpretación final de los resultados obtenidos y validación de la afirmación del fabricante.

\section{Referencias}
\printbibliography

\end{document}
